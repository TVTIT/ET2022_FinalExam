\section*{LỜI NÓI ĐẦU}
\thispagestyle{empty}
Trong kỷ nguyên công nghệ số hiện nay, việc áp dụng các giải pháp thông minh vào quản lý và vận hành trong các lĩnh vực như giáo dục đã trở thành xu thế tất yếu. Một trong những nhu cầu cấp thiết của các trường học, đặc biệt là các trường đại học, là quản lý hiệu quả việc điểm danh sinh viên, đảm bảo tính chính xác, nhanh chóng và minh bạch trong theo dõi sự tham gia học tập.

Dự án Thiết bị và Phần mềm điểm danh bằng thẻ sinh viên được thực hiện nhằm giải quyết bài toán này, mang đến một giải pháp tích hợp giữa phần cứng và phần mềm, giúp tự động hóa quy trình điểm danh và giảm thiểu các thao tác thủ công. Sản phẩm hướng đến việc cải thiện trải nghiệm sử dụng của giảng viên và sinh viên, đồng thời tăng cường tính bảo mật và chính xác trong việc quản lý thông tin.

Dự án là sự kết hợp giữa lập trình phần mềm và thiết kế phần cứng, sử dụng các công nghệ hiện đại như vi điều khiển Arduino, ESP8266, giao tiếp RFID với thẻ sinh viên, và phần mềm quản lý thông tin trên máy tính. Thông qua đó, sản phẩm không chỉ là một bài tập kỹ thuật mà còn là một ứng dụng thực tiễn có giá trị trong môi trường học đường.
\cleardoublepage